%%%%%%%%%%%%%%%%%%%%%%%%%%%%%%%%%%%%%%%%%
% Structured General Purpose Assignment
% LaTeX Template
%
% This template has been downloaded from:
% http://www.latextemplates.com
%
% Original author:
% Ted Pavlic (http://www.tedpavlic.com)
%
% Note:
% The \lipsum[#] commands throughout this template generate dummy text
% to fill the template out. These commands should all be removed when
% writing assignment content.
%
%%%%%%%%%%%%%%%%%%%%%%%%%%%%%%%%%%%%%%%%%

%----------------------------------------------------------------------------------------
%	PACKAGES AND OTHER DOCUMENT CONFIGURATIONS
%----------------------------------------------------------------------------------------

\documentclass{article}

\usepackage{multirow}
\usepackage{amssymb}
\usepackage[fleqn]{amsmath}
\usepackage{url}

\usepackage{fancyhdr} % Required for custom headers
\usepackage{lastpage} % Required to determine the last page for the footer
\usepackage{extramarks} % Required for headers and footers
\usepackage{graphicx} % Required to insert images
\usepackage{lipsum} % Used for inserting dummy 'Lorem ipsum' text into the template

% Margins
\topmargin=-0.45in
\evensidemargin=0in
\oddsidemargin=0in
\textwidth=6.5in
\textheight=9.0in
\headsep=0.25in

\linespread{1.1} % Line spacing

% Set up the header and footer
\pagestyle{fancy}
\lhead{\hmwkAuthorName} % Top left header
\chead{\hmwkClass\ : \hmwkTitle} % Top center header
\rhead{\firstxmark} % Top right header
\lfoot{\lastxmark} % Bottom left footer
\cfoot{} % Bottom center footer
\rfoot{Page\ \thepage\ of\ \pageref{LastPage}} % Bottom right footer
\renewcommand\headrulewidth{0.4pt} % Size of the header rule
\renewcommand\footrulewidth{0.4pt} % Size of the footer rule

\setlength\parindent{0pt} % Removes all indentation from paragraphs

%----------------------------------------------------------------------------------------
%	DOCUMENT STRUCTURE COMMANDS
%	Skip this unless you know what you're doing
%----------------------------------------------------------------------------------------

% Header and footer for when a page split occurs within a problem environment
\newcommand{\enterProblemHeader}[1]{
\nobreak\extramarks{#1}{#1 continued on next page\ldots}\nobreak
\nobreak\extramarks{#1 (continued)}{#1 continued on next page\ldots}\nobreak
}

% Header and footer for when a page split occurs between problem environments
\newcommand{\exitProblemHeader}[1]{
\nobreak\extramarks{#1 (continued)}{#1 continued on next page\ldots}\nobreak
\nobreak\extramarks{#1}{}\nobreak
}

\setcounter{secnumdepth}{0} % Removes default section numbers
\newcounter{homeworkProblemCounter} % Creates a counter to keep track of the number of problems

\newcommand{\homeworkProblemName}{}
\newenvironment{homeworkProblem}[1][Problem \arabic{homeworkProblemCounter}]{ % Makes a new environment called homeworkProblem which takes 1 argument (custom name) but the default is "Problem #"
\stepcounter{homeworkProblemCounter} % Increase counter for number of problems
\renewcommand{\homeworkProblemName}{#1} % Assign \homeworkProblemName the name of the problem
\section{\homeworkProblemName} % Make a section in the document with the custom problem count
\enterProblemHeader{\homeworkProblemName} % Header and footer within the environment
}{
\exitProblemHeader{\homeworkProblemName} % Header and footer after the environment
}

\newcommand{\problemAnswer}[1]{ % Defines the problem answer command with the content as the only argument
\noindent\framebox[\columnwidth][c]{\begin{minipage}{0.98\columnwidth}#1\end{minipage}} % Makes the box around the problem answer and puts the content inside
}

\newcommand{\homeworkSectionName}{}
\newenvironment{homeworkSection}[1]{ % New environment for sections within homework problems, takes 1 argument - the name of the section
\renewcommand{\homeworkSectionName}{#1} % Assign \homeworkSectionName to the name of the section from the environment argument
\subsection{\homeworkSectionName} % Make a subsection with the custom name of the subsection
\enterProblemHeader{\homeworkProblemName\ [\homeworkSectionName]} % Header and footer within the environment
}{
\enterProblemHeader{\homeworkProblemName} % Header and footer after the environment
}

%----------------------------------------------------------------------------------------
%	NAME AND CLASS SECTION
%----------------------------------------------------------------------------------------


%----------------------------------------------------------------------------------------
%	TITLE PAGE
%----------------------------------------------------------------------------------------

\title{
\vspace{2in}
\textmd{\textbf{\hmwkClass:\ \hmwkTitle}}\\
\normalsize\vspace{0.1in}\small{Due\ on\ \hmwkDueDate}\\
\vspace{3in}
}

\author{\textbf{\hmwkAuthorName}}
\date{} % Insert date here if you want it to appear below your name



\usepackage{clrscode}
\usepackage{enumerate}
\newcommand{\hmwkTitle}{Assignment\ \#5} % Assignment title
\newcommand{\hmwkDueDate}{April\ 4,\ 2016} % Due date
\newcommand{\hmwkClass}{Algorithms} % Course/class
\newcommand{\hmwkAuthorName}{Zhaoyang Li (2014013432)} % Your name
%----------------------------------------------------------------------------------------

\begin{document}

\maketitle

%----------------------------------------------------------------------------------------
%	TABLE OF CONTENTS
%----------------------------------------------------------------------------------------

\setcounter{tocdepth}{1} % Uncomment this line if you don't want subsections listed in the ToC

\newpage
\tableofcontents
\newpage

%----------------------------------------------------------------------------------------
%	PROBLEM 1
%----------------------------------------------------------------------------------------


\begin{homeworkProblem}

Read CLRS chapters 12, 13 and 18.

\problemAnswer{
}

\end{homeworkProblem}


%----------------------------------------------------------------------------------------
%	PROBLEM 2
%----------------------------------------------------------------------------------------


\begin{homeworkProblem}

CLRS Exercises 17.4-3.


\problemAnswer{

Let $c_i$ be actual cost of deletion operation $i$, and $\hat{c_i}$ be the corresponding amortized one. Let $n_{i-1}, s_{i-1}, n_i, s_i$ be number of elements in the table and size of the table, before and after the deletion, respectively.

When a deletion happens without resizing,

$$
s_i = s_{i-1},
n_i = n_{i-1} - 1,
n_i p\geq \frac13 s_{i-1},
$$

Therefore

\begin{equation}
\begin{aligned}
\hat{c_i}&=c_i+\Phi_i-\Phi_{i-1} \\
&=1+|2 n_i - s_i| - |2 n_{i-1} - s_{i-1}| \\
&\leq1+|\left(2 \left(n_{i-1}-1\right) - s_{i-1}\right) - \left(2 n_{i-1} - s_{i-1}\right)| \\
&=1+2\\
&=3.
\end{aligned}
\end{equation}

When a resizing happens, we have

$$
s_i = \lfloor\frac23 s_{i-1}\rfloor,
n_i = n_{i-1} - 1,
n_{i-1} \geq\frac13 s_{i-1},
n_{i} < \frac13 s_{i-1}
$$

Therefore
\begin{equation}
\begin{aligned}
\hat{c_i}&=c_i+\Phi_i-\Phi_{i-1} \\
&=\left(n_i+1\right) +|2 n_i - s_i| - |2 n_{i-1} - s_{i-1}| \\
&=\left(n_i+1\right) + |s_i - 2 n_i| -\left(-2 n_{i-1} + s_{i-1}\right)\\
&\leq\left(n_i+1\right) + 2 - n_i \\
&=3.
\end{aligned}
\end{equation}

Therefore

$$\hat{c_i}=O(1)$$

QED.
}

\end{homeworkProblem}

%----------------------------------------------------------------------------------------
%	PROBLEM 3
%----------------------------------------------------------------------------------------


\begin{homeworkProblem}

CLRS Problems 17-2.
Making binary search dynamic.

a.

\problemAnswer{
To search, perform standard binary search on each of the $k$ sorted arrays.

Worst-case running time
\begin{equation}
\begin{aligned}
T(n)
& \leq\sum_{i=0}^{k-1}O(\lg{2^i}) \\
& =\sum_{i=0}^{k-1}O(i) \\
& =O\left(\sum_{i=0}^{k-1}i\right) \\
& =O\left(k^2\right) \\
& =O\left(\lceil\lg(n+1)\rceil^2\right)\\
& =O\left(\lg^2n\right)
\end{aligned}
\end{equation}
}

b.

\problemAnswer{
To insert an element $e$, 
\begin{quote}
\begin{enumerate}[Step 1.]
\item Find the max $j$ such that $\forall i\ s.t\ 0\leq i<j, n_i = 1$ and $n_j=0$. If such $j$ does not exist, we say $j=0$.
\item Let $A_j=\bigcup_{i=0}^{j-1}A_{i}$. 
\item Insert $e$ into $A_j$.
\item For all $i$ such that $0\leq i < j$, let $A_{i}=\langle\rangle, n_i=0$.
\item Let $n_j = 1, n = n+1$. Done.
\end{enumerate}
\end{quote}

Timing:\\ \\ 
Steps 1, 6 take $O(1)$ time. Step 3 takes time linear to length of $A_j$, which is $O(2^j)$. Step 4 takes $O(j)$. Step 2 involves merging sorted arrays and can be done in $O(2^j)$ time. The total is $O(2^j)$.
\\ \\ 
In the worst case, $j=k-1$, meaning $n=2^k-1$. Running time $O(2^j)=O(2^k)=O(2^{\lg n}) = O(n)$.
\\ \\ 
During $m$ insertion operations, noticing the nature of binary increment, in $\frac m2$ cases we have $j=0$, in $\frac m4$ for $j=1$, $\frac m8$ for $j=2$... In a word, we know the distribution of $j$.

$$T_m
\leq\sum_{i=0}^{k-1}\frac{m}{2^i} O(\lg{2^{i+1}}) 
\leq\sum_{i=0}^{k-1}m O(1)
=mO(k)
=mO(\lg n)$$

Applying aggregate analysis, amortized running time for each operation 



$$T = \frac{m O(\lg n)}{m} = O(\lg n)$$
}

c.

\problemAnswer{
To perform a deletion of element $e$, we first perform a search to find $i$ s.t. $e\in A_i$. Remove $e$ from $A_i$. 

Also, we find the minimal $j$ s.t. $A_j\neq \langle\rangle$. Take an element $a\in A_j$ randomly and insert it into $A_i$. Now forget about $A_i$ and look at $A_j$, whose length is $2^j-1$. We divide $A_j$ into arrays $A_0, A_1, ..., A_{j-1}$ who have a total capacity of $2^j-1$ and are by now empty. Done.
}
\end{homeworkProblem}


%----------------------------------------------------------------------------------------
%	PROBLEM 4
%----------------------------------------------------------------------------------------


\begin{homeworkProblem}

CLRS Problems 19-3. More Fibonacci-heap operations.

a.

\problemAnswer{
If $k < x.key$, we call \proc{Fib-Heap-Decrease-Key}$(H, x, k)$, amortized cost of which is O(1).

If $k = x.key$, we do nothing and return, with $O(1)$ time consumption.


If $k > x.key$, call \proc{Cut}$(H, y, y.p)$ for each child of $x$ such that $y.p = x$. Then let $x.key = k$, call \proc{Cut}$(H, x, x.p)$. Done. Since $\forall z \in H, z.degree=O(\lg n)$, we have the amortized cost $O(\lg n)$.
}

b.

\problemAnswer{
Delete leaves only. We may want to keep track of all the leaves, so that deletion of a single leaf takes $O(1)$ time.

Let $\Phi(D_i)=t(H) + m(H)$.

$\hat{c_i} = c_i +  \Phi(D_i) - \Phi(D_{i-1})
= q O(1) - q
= O(q)
$.
}
\end{homeworkProblem}



%----------------------------------------------------------------------------------------

\end{document}
